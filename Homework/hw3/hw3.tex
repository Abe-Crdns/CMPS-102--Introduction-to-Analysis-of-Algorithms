\documentclass[11pt]{article}
\usepackage{fullpage,amsthm,amsfonts,amssymb,epsfig,amsmath,times,amsthm}
\usepackage{algpseudocode}
\usepackage{xcolor}
\usepackage{hyperref}

\hypersetup{
    colorlinks=true,
    urlcolor=cyan,
}
\urlstyle{same}

\newtheorem{theorem}{Theorem}
\newtheorem{claim}[theorem]{Claim}

\renewcommand{\blacksquare}{\textcolor{blue}{\openbox}}

\begin{document}

\textcolor{blue}{\hfill Abraham Cardenas} 


\begin{center}
{\bf\Large CMPS 102 --- Winter 2019 --  Homework 3}
\end{center}

\newcommand{\Mod}[1]{\ \mathrm{mod}\ #1}

\begin{itemize}

\item[$1.$] (9 pts)

a.

\textcolor{blue}{ 
\underline{Proof by contradiction}:\\
Suppose we had the minimum spanning tree $T$ with the set of edge weights $w_1, w_2,$ .... $w_n$.  Suppose $T$ was not a minimum-altitude connected subgraph. There should be a pair of nodes $u$ and $v$, and two $u$ to $v$ paths $P_1 \neq P_2$, such that $P_1$ is the path from $u$ to $v$ but $P_2$ has the smaller altitude. Then, there exists some edge $e = (u',v')$ on $P_1$ that contains the maximum altitude over all edges in $P_1$ and $P_2$.\\ 
Now, let's consider the set of edges in $P_1$ and $P_2$ except $e$, and suppose we took the following path from $u'$-$v'$:
\begin{enumerate}
  \item $P_1$ from $u'$ to $u$
  \item $P_2$ from $u$ to $v$
  \item $P_1$ from $v$ to $v'$
\end{enumerate}
It's possible that this path from $u'$-$v'$ contains a cycle so then, the set of edges in $P_1$ and $P_2$ except $e$ must contain some simple path $S$. And because $e$ is the maximum altitude in $T$, this would imply that the edges in $S$ and $e$ would also form a cycle. This contradicts the Cycle Property which says that ``for any cycle C in the graph, if the weight of an edge e of C is larger than the individual weights of all other edges of C, then this edge cannot belong to an MST''. Therefore, we can conclude that $T$ must be a minimum-altitude connected subgraph.\hspace{7.5cm}\blacksquare
}


\item[$2.$] (6 pts)\\
a.\\
\textcolor{blue}{ 
If we choose $\omega = 3$, then we have $$\Rightarrow \big\{ \omega^0, \omega^1, \omega^2, \omega^3, \omega^4, \omega^5  \big\} = \big\{ 3^0,  3^1, 3^2, 3^3, 3^4, 3^5  \big\}$$ $$\Rightarrow \big\{ 1, 3, 9, 27, 81, 243  \big\}$$ In mod 7 we have, $$= \big\{ 1, 3, 2, 6, 4, 5  \big\}$$
}
b.\\
\textcolor{blue}{ 
 \[
   F=
  \left[ {\begin{array}{cccccc}
   1 & 1 & 1 & 1 & 1 & 1\\
   1 & 3 & 2 & 6 & 4 & 5\\
   1 & 2 & 4 & 1 & 2 & 4\\
   1 & 6 & 1 & 6 & 1 & 6\\
   1 & 4 & 2 & 1 & 4 & 2\\
   1 & 5 & 4 & 6 & 2 & 3\\
  \end{array} } \right]
\]
}
c.\\
\textcolor{blue}{ 
$$\textbf{y} = F \cdot \textbf{a}$$
 \[
   \Rightarrow \hspace{1.5cm} \textbf{y} =
  \left[ {\begin{array}{cccccc}
   1 & 1 & 1 & 1 & 1 & 1\\
   1 & 3 & 2 & 6 & 4 & 5\\
   1 & 2 & 4 & 1 & 2 & 4\\
   1 & 6 & 1 & 6 & 1 & 6\\
   1 & 4 & 2 & 1 & 4 & 2\\
   1 & 5 & 4 & 6 & 2 & 3\\
  \end{array} } \right]
  \cdot 
    \left[ {\begin{array}{c}
   0\\
   1\\
   2\\
   1\\
   5\\
   2\\
  \end{array} } \right]
   	= 
	\left[ {\begin{array}{c}
  	 11\\
 	  43\\
  	 29\\
  	 31\\
  	 33\\
  	 35\\
  \end{array} } \right]
    \]
In mod 7 we have, 
   \[
	\textbf{y} =
      \left[ {\begin{array}{c}
   4\\
   1\\
   1\\
   3\\
   5\\
   0\\
  \end{array} } \right]
\]
}
d.\\
\phantom{d.} i.
\textcolor{blue}{ 
 \[
   G_n = \frac{1}{6} \cdot
  \left[ {\begin{array}{cccccc}
   1 & 1 & 1 & 1 & 1 & 1\\
   1 & 3^{-1} & 3^{-2} & 3^{-3} & 3^{-4} & 3^{-5}\\
   1 & 3^{-2} & 3^{-4} & 3^{-6} & 3^{-8} & 3^{-10}\\
   1 & 3^{-3} & 3^{-6} & 3^{-9} & 3^{-12} & 3^{-15}\\
   1 & 3^{-4} & 3^{-8} & 3^{-12} & 3^{-16} & 3^{-20}\\
   1 & 3^{-5} & 3^{-10} & 3^{-15} & 3^{-20} & 3^{-25}\\
  \end{array} } \right]
\]
 \[
  = \frac{1}{6} \cdot
   \left[ {\begin{array}{cccccc}
   1 & 1 & 1 & 1 & 1 & 1\\
   1 & 3^{-1} & 2^{-1} & 6^{-1} & 4^{-1} & 5^{-1}\\
   1 & 2^{-1} & 4^{-1} & 1^{-1} & 2^{-1} & 4^{-1}\\
   1 & 6^{-1} & 1^{-1} & 6^{-1} & 1^{-1} & 6^{-1}\\
   1 & 4^{-1} & 2^{-1} & 1^{-1} & 4^{-1} & 2^{-1}\\
   1 & 5^{-1} & 4^{-1} & 6^{-1} & 2^{-1} & 3^{-1}\\
  \end{array} } \right]
  \]
In mod 7 we have,
$$1^{-1} = 1, 2^{-1} = 4, 3^{-1} = 5, 4^{-1} = 2, 5^{-1} = 3, 6^{-1} = 6$$
  \[
 	\Rightarrow 6 \cdot
 \left[ {\begin{array}{cccccc}
   1 & 1 & 1 & 1 & 1 & 1\\
   1 & 5 & 4 & 6 & 2 & 3\\
   1 & 4 & 2 & 1 & 4 & 2\\
   1 & 6 & 1 & 6 & 1 & 6\\
   1 & 2 & 4 & 1 & 2 & 4\\
   1 & 3 & 2 & 6 & 4 & 5\\
  \end{array} } \right]
  = 
   \left[ {\begin{array}{cccccc}
   6 & 6 & 6 & 6 & 6 & 6\\
   6 & 30 & 24 & 36 & 12 & 18\\
   6 & 24 & 12 & 6 & 24 & 12\\
   6 & 36 & 6 & 36 & 6 & 36\\
   6 & 12 & 24 & 6 & 12 & 24\\
   6 & 18 & 12 & 36 & 24 & 30\\
  \end{array} } \right]  
\]
Again, in mod 7 we have, 
 \[
 G_n =
   \left[ {\begin{array}{cccccc}
   6 & 6 & 6 & 6 & 6 & 6\\
   6 & 2 & 3 & 1 & 5 & 4\\
   6 & 3 & 5 & 6 & 3 & 5\\
   6 & 1 & 6 & 1 & 6 & 1\\
   6 & 5 & 3 & 6 & 5 & 3\\
   6 & 4 & 5 & 1 & 3 & 2\\
  \end{array} } \right]
  \]
}
\phantom{d.} ii.
\textcolor{blue}{ 
$$\textbf{a} = G_n \cdot \textbf{y}$$
 \[
 \Rightarrow \hspace{1.5cm} G_n \cdot \textbf{y} =
   \left[ {\begin{array}{cccccc}
   6 & 6 & 6 & 6 & 6 & 6\\
   6 & 2 & 3 & 1 & 5 & 4\\
   6 & 3 & 5 & 6 & 3 & 5\\
   6 & 1 & 6 & 1 & 6 & 1\\
   6 & 5 & 3 & 6 & 5 & 3\\
   6 & 4 & 5 & 1 & 3 & 2\\
  \end{array} } \right]
  \cdot
   \left[ {\begin{array}{c}
   4\\
   1\\
   1\\
   3\\
   5\\
   0\\
  \end{array} } \right]
   	 =
     \left[ {\begin{array}{c}
   84\\
   57\\
   65\\
   64\\
   75\\
   51\\
  \end{array} } \right]
  \hspace{2cm}
    \]
In mod 7 we have,
  \[
\textbf{a} =
  \left[ {\begin{array}{c}
   0\\
   1\\
   2\\
   1\\
   5\\
   2\\
  \end{array} } \right]
\]
}

\end{itemize}

\begin{center}
Sources
\end{center}

\url{https://en.wikipedia.org/wiki/Minimum_spanning_tree#Cycle_property}

\end{document}